% qjrms4doc.tex V1.10, 4 October 2013

\documentclass[times]{qjrms4}

\usepackage[colorlinks,bookmarksopen,bookmarksnumbered,citecolor=red,urlcolor=red]{hyperref}

\newcommand\BibTeX{{\rmfamily B\kern-.05em \textsc{i\kern-.025em b}\kern-.08em
T\kern-.1667em\lower.7ex\hbox{E}\kern-.125emX}}

\usepackage{moreverb}

\def\volumeyear{2013}
%\def\volumenumber{00}

\begin{document}

\runningheads{A.~N.~Other}{A demonstration of the \emph{Q.~J.~R.
Meteorol. Soc.} class file}

\title{A demonstration of the \LaTeXe\ class file for
the \itshape{Quarterly Journal of the Royal Meteorological
Society}\footnotemark[2]}

\author{A.~N.~Other\corrauth}

\address{John Wiley \& Sons, Ltd, The Atrium, Southern Gate, Chichester,
West Sussex, PO19~8SQ, UK}

\corraddr{Journals Production Department, John Wiley \& Sons, Ltd,
The Atrium, Southern Gate, Chichester, West Sussex, PO19~8SQ, UK.}

\begin{abstract}
This paper describes the use of the \LaTeXe\ \textsf{qjrms4.cls}
class file for setting papers for the \emph{Quarterly Journal of
the Royal Meteorological Society}.
\end{abstract}

\keywords{class file; \LaTeXe; \emph{Q.~J.~R. Meteorol. Soc.}}

\maketitle

\footnotetext[2]{Please ensure that you use the most up to date
class file,
available from the QJRMS Home Page at\\
\href{http://onlinelibrary.wiley.com/journal/10.1002/(ISSN)1477-870X}{\tiny\texttt{http://onlinelibrary.wiley.com/journal/10.1002/(ISSN)1477-870X}}%
}

\section{Introduction} Modification de l introduction

\section{The three golden rules} Before we proceed, we would like to
stress \emph{three golden rules} that need to be followed to
enable the most efficient use of your code at the typesetting
stage:
\begin{enumerate}
\item[(i)] keep your own macros to an absolute minimum;
\item[(ii)] as \TeX\ is designed to make sensible spacing
decisions by itself, do \emph{not} use explicit horizontal or
vertical spacing commands, except in a few accepted (mostly
mathematical) situations, such as \verb"\," before a
differential~d, or \verb"\quad" to separate an equation from its
qualifier;
\item[(iii)] follow the \emph{Quarterly Journal}
reference style.
\end{enumerate}

\section{Getting started} The \textsf{qjrms4} class file should run
on any standard \LaTeXe\ installation. If any of the fonts, class
files or packages it requires are missing from your installation,
they can be found on the \emph{\TeX\ Collection} DVDs or from
CTAN.

The \emph{Quarterly Journal} is published using a proprietary
font. A reasonable match can be achieved by using the Times fonts
and this is achieved by using the \verb"times" option as
\verb"\documentclass[times]{qjrms4}". If for any reason you have a
problem using Times you can easily resort to Computer Modern fonts
by removing the \verb"times" option.

\section{The article header information}
The heading for any file using \textsf{qjrms4.cls} is shown in
Figure~\ref{F1}.

\begin{figure*}
\setlength{\fboxsep}{0pt}%
\setlength{\fboxrule}{0pt}%
\begin{center}
\begin{boxedverbatim}
\documentclass[times]{qjrms4}
%\documentclass[times,doublespace]{qjrms4}%For paper submission

\begin{document}

\runningheads{<Initials and Surnames>}{<Short title>}

\title{<Initial cap, lower case>}

\author{<An Author,\affil{a}
Someone Else\affil{b}\corrauth\ and Perhaps Another\affil{a}>}

\address{<\affilnum{a}First author's address
(in this example it is the same as the third author)\\
\affilnum{b}Second author's address>}

\corraddr{<Corresponding author's address (the second author in
this example)>. E-mail: <corresponding author's email address>}

\begin{abstract}
<Text>
\end{abstract}

\keywords{<List up to eight key words>}

\maketitle

\section{Introduction}
.
.
.
\end{boxedverbatim}
\end{center}
\caption{Example header text.\label{F1}}
\end{figure*}

\subsection{Remarks}
\begin{enumerate}
\item[(i)] In \verb"\runningheads", keep the short title to no
more than 50 characters; use `\emph{et~al.}' if there are three or
more authors.

\item[(ii)] Note the use of \verb"\affil" and \verb"\affilnum" to
link names and addresses. The author for correspondence is marked
by \verb"\corrauth" and \verb"\corraddr" is used to give that
author's address, which will be printed prefaced by
`Correspondence to:'.

\item[(iii)] For submitting a double-spaced manuscript, add
\verb"doublespace" as an option to the documentclass line.

\item[(iv)] Keywords are separated by semicolons.
\end{enumerate}

\section{The body of the article}

\subsection{Mathematics} \textsf{qjrms4.cls} makes the full
functionality of \AmS\/\TeX\ available. We encourage the use of
the \verb"align", \verb"gather" and \verb"multline" environments
for displayed mathematics.

\subsection{Figures and tables} \textsf{qjrms4.cls} uses the
\textsf{graphicx} package for handling figures.

Figures are called in as follows:
\begin{verbatim}
\begin{figure}
\centering
\includegraphics{<figure name>}
\caption{<Figure caption>}
\end{figure}
\end{verbatim}
Recall that the\\
\verb"\begin{figure*}...\end{figure*}"\\
commands are needed for a
figure spanning both columns.

For further details on how to size figures, etc, with the
\textsf{graphicx} package see, for example, \cite{R1} or
\cite{R3}. If figures are available in an acceptable format (for
example, .eps, .ps) they will be used but a printed version should
always be provided. \medbreak

The standard coding for a table is shown in Figure~\ref{F2}.

\begin{figure}
\begin{verbatim}
\begin{table}
\caption{<Table caption>}
\centering
\begin{tabular}{<table alignment>}
\toprule
<column headings>\\
\midrule
<table entries
(separated by & as usual)>\\
<table entries>\\
.
.
.\\
\bottomrule
\end{tabular}
\end{table}
\end{verbatim}
\caption{Example table layout.\label{F2}}
\end{figure}

\subsection{Cross-referencing}
The use of the \LaTeX\ cross-reference system
for figures, tables, equations, etc, is encouraged
(using \verb"\ref{<name>}" and \verb"\label{<name>}").

\subsection{Acknowledgements}
An Acknowledgements section is started with \verb"\ack" or
\verb"\acks" for \textit{Acknowledgement} or
\textit{Acknowledgements}, respectively. It must be placed just
before the References (or before the appendix when applicable).

\subsection{Bibliography}
Please note that the file \textsf{wileyqj.bst} is available from
the same download page for those authors using \BibTeX.

Otherwise, the normal commands for producing the reference list
are:
\begin{verbatim}
\begin{thebibliography}{99}
\item <Reference details>
.
.
.
\item <Reference details>
\end{thebibliography}
\end{verbatim}

\subsection{Double spacing}
If you need to double space your document for submission please
use the \verb+doublespace+ option as shown in the sample layout in
Figure~\ref{F1}.

\section{Support for \textsf{qjrms4.cls}}
We offer on-line support to participating authors.
Please contact us via e-mail at
\href{mailto:qjauth-cls@wiley.co.uk}{\texttt{qjauth-cls@wiley.co.uk}}.

We would welcome any feedback, positive or otherwise, on your
experiences of using \textsf{qjrms4.cls}.

\section{Copyright statement}
Please  be  aware that the use of  this \LaTeXe\ class file is
governed by the following conditions.

\subsection{Copyright}
Copyright \copyright\ \volumeyear\ John Wiley \& Sons, Ltd, The Atrium,
Southern Gate, Chichester, West Sussex, PO19~8SQ, UK.  All rights reserved.

\subsection{Rules of use}
This class file is made available for use by authors who wish to
prepare an article for publication in the \emph{Quarterly Journal
of the Royal Meteorological Society} published by John Wiley \&
Sons, Ltd.  The user may not exploit any part of the class file
commercially.

This class file is provided on an \emph{as~is}  basis, without
warranties of any kind, either express or implied, including but
not limited to warranties of title, or implied  warranties of
merchantablility or fitness for a particular purpose. There will
be no duty on the author[s] of the software or  John Wiley \&
Sons, Ltd to correct any errors or defects in the software. Any
statutory  rights you may have remain unaffected by your
acceptance of these rules of use.

\ack This class file was developed by Sunrise Setting Ltd,
Paignton, Devon, UK. Website:\\
\href{http://www.sunrise-setting.co.uk}{\texttt{www.sunrise-setting.co.uk}}

\begin{thebibliography}{99}
\bibitem[Kopka and Daly(2003)]{R1}
Kopka~H, Daly~PW. 2003. \emph{A Guide to \LaTeX} (4th~edn).
Addison-Wesley.

\bibitem[Lamport(1994)]{R2}
Lamport~L. 1994. \emph{\LaTeX: a Document Preparation System}
(2nd~edn). Addison-Wesley.

\bibitem[Mittelbach and Goossens(2004)]{R3}
Mittelbach~F, Goossens~M. 2004. \emph{The \LaTeX\ Companion}
(2nd~edn). Addison-Wesley.

\end{thebibliography}
\end{document}

